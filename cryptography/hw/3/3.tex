\documentclass[12pt,border=4pt,multi]{article} % \documentclass[tikz,border=4pt,multi]{article}
\usepackage{lingmacros}
\usepackage{tree-dvips}
\usepackage{amssymb} % for mathbb{}
\usepackage[dvipsnames]{xcolor}
\usepackage{forest}
\usepackage{amsmath} % for matrices
\usepackage{xeCJK}
\usepackage{tikz}
\usepackage[arrowdel]{physics}
\usepackage{graphicx}
\usepackage{wrapfig}
\usepackage{listings}
\usepackage{pgfplots, pgfplotstable}
\usepackage{diagbox} % diagonal line in cell
\usepackage[usestackEOL]{stackengine}
\usepackage{multirow}
\graphicspath{{./img}} % specify the graphics path to be relative to the main .tex file, denoting the main .tex file directory as ./
\newcommand\Myperm[2][^n]{\prescript{#1\mkern-2.5mu}{}P_{#2}}
\newcommand\Mycomb[2][^n]{\prescript{#1\mkern-0.5mu}{}C_{#2}}
\definecolor{orchid}{rgb}{0.7, 0.4, 1.1}

\begin{document}
\section*{Xi Liu, Assignment 3}
1\\
if it is easy to find a $s2$ such that $H(s1) = H(s2)$, then it is possible to carry out a classical collision attack. since bitcoins are rewarded to the public key when the hash value satisfies proof of work requirement ($k$ bits are 0), then if $H(s1)$ meets proof of work requirement, then $H(s2)$ also meets proof of work requirement, then 2 mining rewards are given to the same hash value\\
\\
\\
\\
\\
2\\
1) no, it can be detected, since all transactions are recorded in block chain which is a shared public ledger\\
2) suppose $pk_i$ is the public key of victim, $pk_j$ is public key of receiver\\
adversary can steal bitcoins by using the forged signature to send money from $pk_i$ to $pk_j$ in the form $sign(pk_i\;\underrightarrow{x\;coins}\;pk_j)$. the transaction succeeds since the validity of the transaction is checked through 2 criteria: 1. valid signature, 2. $pk_i$ would not overdrawn after send. then the bitcoins cannot be retrieved to the victim since transactions are irreversible due to the append only nature of block chain public ledger\\
\\
\\
\\
\\
3\\
it is not secure if the who mines the current block pick a very easy DNA sequencing problem for the next miner to solve, then the next miner can obtain the bitcoin rewards in a very short time\\
\\
\\
\\
\\
4\\
it is possible but very unlikely for the forks to remain to be equally long, since people have different computing power and the probability that mining in a single attempt for a hash output that have $k$ bits of 0 is approximately $1 / 2^k$\\
when 1 of the branches is merely 1 block longer than the other, people tend to continue to mine on the slightly longer block since it is less likely to be erased\\ 
\\
\\
\\
\\
5\\
let $cdh$ be the given algorithm that solves the computational Diffie Hellman problem such that $g^{ab} \leftarrow cdh(g, g^a, g^b)$\\
construction of Elgamal scheme:\\
let $G_p$ be a group that satisfies decisional Diffie Hellman assumption, $g \in G_q$ be a generator
\begin{lstlisting}[mathescape = true]
$gen(n)$
{
    sample random element $x$ from $\mathbb{Z}_q^*$
    return $(pk = g^x, sk = x)$
}
$enc(pk, m)$
{
    sample random element $r$ from $\mathbb{Z}_q^*$
    return $(c_1 = g^r, c_2 = pk^r \cdot m)$
}
\end{lstlisting}
to break Elgamal encryption, adversary need to find message $m$ without knowing the secret key $sk$. assumes adversary only knows $pk = g^x$, $c_1 = g^r$, and $c_2 = pk^r \cdot m$\\
\begin{align*}
pk^r &= (g^x)^r = g^{xr}\\
\text{use the $cdh()$ algorithm: } & g^{xr} \leftarrow cdh(g, g^x, g^r)\\
\text{now adversary knows $pk^r = g^{xr}$ and $c_2$} &\\
c_2 &= pk^r \cdot m\\
m &= c_2 \cdot pk^{-r}\\
\end{align*}
\newpage
\noindent
6\\
suppose $m$ is Bob's bid (i.e., $m = 100$), since Trudy knows $(c_1 = g^r, c_2 = pk^r \cdot m) \leftarrow enc(pk, m)$, Trudy can produce a ciphertext $c_2'$ such that $c_2' = 2 \cdot c_2 = 2(pk^r \cdot m) = pk^r \cdot (2m)$
\end{document}
